%%%%%%%%%%%%%%%%%%%%%%%%%%%%%%%%%%%%%%%%%
% Journal Article
% LaTeX Template
% Version 2.0 (February 7, 2023)
%
% This template originates from:
% https://www.LaTeXTemplates.com
%
% Author:
% Vel (vel@latextemplates.com)
%
% License:
% CC BY-NC-SA 4.0 (https://creativecommons.org/licenses/by-nc-sa/4.0/)
%
% NOTE: The bibliography needs to be compiled using the biber engine.
%
%%%%%%%%%%%%%%%%%%%%%%%%%%%%%%%%%%%%%%%%%

%----------------------------------------------------------------------------------------
%	PACKAGES AND OTHER DOCUMENT CONFIGURATIONS
%----------------------------------------------------------------------------------------

\documentclass[
	a4paper, % Paper size, use either a4paper or letterpaper
	10pt, % Default font size, can also use 11pt or 12pt, although this is not recommended
	unnumberedsections, % Comment to enable section numbering
	twoside, % Two side traditional mode where headers and footers change between odd and even pages, comment this option to make them fixed
]{LTJournalArticle}

%\addbibresource{sample.bib} % BibLaTeX bibliography file

\runninghead{Technology as Magic } % A shortened article title to appear in the running head, leave this command empty for no running head

\footertext{\textit{Deus Ex Machina Conference} (October 24, 2025) } % Text to appear in the footer, leave this command empty for no footer text

\setcounter{page}{1} % The page number of the first page, set this to a higher number if the article is to be part of an issue or larger work

%----------------------------------------------------------------------------------------
%	TITLE SECTION
%----------------------------------------------------------------------------------------

\title{Technology as Magic, A Story of Errors}   % Article title, use manual lines breaks (\\) to beautify the layout

% Authors are listed in a comma-separated list with superscript numbers indicating affiliations
\author{Paul Heath}
\date{\today}


% Affiliations are output in the \date{} command
\date{\footnotesize\textsuperscript{\textbf{1}}Seagate Technologies LLC } 
 

% Full-width abstract
\renewcommand{\maketitlehookd}{%
\begin{abstract}
\noindent 
Technological advances tend to appear as something magical when they are first introduced. 
An aspect of all technology that may not be considered in any depth, is how systems respond to 
error conditions. Often, the very essence of the technological solution is in how errors are 
handled. 
In this paper, several areas of technology are covered. Each one has different uses and applications. 
In each example, error handling is either a fundamental limit to the performance of the system, or it 
is a controlled parameter that is central to the behavior of the system. With the recent advent of 
artificial intelligence systems, we have to consider a few new aspects of how systems handle errors. 
AI systems have the ability to at least appear to have a point of view. They have been used to 
purposefully create products and responses that are objectively false or misleading. They also, due to the
nature of a training/classification model, can be trained using data that is incorrect, or has a 
bias of some kind. We discuss the data security concepts of Confidentiality, Integrity, and 
Accessibility, and add a potential concept of Data Validity. This idea can lead to a discussion
of how metrics might arise which enable the classification of data as being valid or not.
\end{abstract}
}

%----------------------------------------------------------------------------------------

\begin{document}

\maketitle % Output the title section

%----------------------------------------------------------------------------------------
%	ARTICLE CONTENTS
%----------------------------------------------------------------------------------------



\section{A History of Technologies}

\subsection{Digital Computing in 1975 }
The mid 1970s was early in the 'big iron' computer era. With IBM having moved from the 
701 machine in the late 1950s to the 370 series in the mid 1970s. Mini-computing was in its early adoption. 
Digital Equipment was leading the way with the PDP series. ATT had a large operating system project 
called MULTICS underway. An alternative operating system was developed by Ken Thompson and Dennis
Ritchie, this famously became Unix. Unix went on to become a dominant factor in both university
and departmental computing. Unix was initially written in PDP-7 assembler language. It went on to 
be refined and rewritten in a new, portable language known as C. C remains an important language today
in low level applications and embedded systems, including its use as the primary language used 
in the Linux operating system, a very popular Unix style system. 

The author's first exposure to computing happened during the mid 1970s. The high school had a class
in computer science. The system in use was a DEC PDP-11 running the RSTS/E operating system. 
Since the CPU was located at a small company, remote from the school, connections were made 
from a Teletype ASR-33 terminal, through a modem. The modem used a standard phone handset, and 
ran at 110 baud, or about 10 characters per second. The interaction with a remote computer looked 
like magic. A command went slowly into a teletype, and in a few seconds, a response came back. 
The limitations on communication speed, computing speed, and mechanical response of the teletype unit 
all combined to determine the response time. As we know today, all of those factors are now orders
of magnitude faster. This is largely due to errors. Communication line speed is set by the rate at
which data can flow with a low error rate. Computer clock rate is limited by silicon feature size, 
and heat transfer rates. These are also driven by errors. This is the first example of how errors
shape our experience of technology.

\subsection{The IBM Storage Subsystem }
IBM pioneered the spinning disk storage device in 1956, with the 305 RAMAC system. This device
used a large spindle stack of disks, and a single magnetic head / arm assembly to move from disk to disk 
within the collection of disks. The entire box stored 5 megabytes and weighed around 2000 pounds.
The introduction of the disk storage concept began an industry that continues today, with disk storage 
being the most cost effective and widely deployed technology in use. By the early 1980s, IBMs disk 
subsystem had matured to what was called the 3380. This device was the first to offer more than 1 gigabyte
of storage per spindle. A box, slightly larger than a refrigerator with a weigh of 1000 pounds, would
store 2.52 gigabytes. In comparison, a modern high capacity disk drive can hold roughly 30 terabytes
of data. Using the baseline of the 3380 device, if cars would have advanced at the same rate as disk drives, a single tank of gas would
be able to take you the distance of 190 times the circumference of the Earth, or to the Moon and 
back 10 times. 

How do errors shape the behavior of disk drives? The first job of a storage device is to save and
recall data perfectly. The device presents a large expanse of 'logical blocks', each being 512 bytes, or 
in some cases 4096 bytes. Each block is numbered. The assumption is that when data is written to 
a given block, it can be read at some later time with exact recall. When a command to read a block
comes down from the host computer, the drive first checks to see if it has that data in its cache 
memory. If so, it sets up a data transfer phase on the interface, and transmits the data to the host. 
If the data is not in the cache, the firmware of the drive tells the 'actuator' containing the 
magnetic heads, to move to the radial location containing the target block. While this is happening,
the firmware sets up the read channel to trigger on the target block and transfer the data into 
cache. If that works, the data is transmitted to the host. If there is a read fault of some kind, 
the firmware goes through a complex sequence of retries, starting with simple re-reads and moving 
into channel equalization tweaks, off track read operations, and short write operations to reset 
magnetic domains in the head material. This process takes many dozens of disk revolutions to 
complete. A great deal of engineering effort has taken place in order to be able to deliver error-free
data to the host device. In the past decade or so, a new use case has emerged. During the read of 
video frame data, a different perspective on errors is used. With video 
streaming, the requirement is that data be available at a rate of up to 25 frames per second at 
the video endpoint. If the disk system is spending dozens of revolutions trying to recover from 
a read error, it is unlikely to be able to keep up with the video frame rate. In this case, it is 
better to deliver data on time, with a few bytes incorrect than it is to deliver perfect data 
at too slow a rate. A single video frame with a few pixels of the wrong color will not be detectable 
by the viewer. The point here is that the definition of error can change with use case, and the
means of dealing with errors has to also change accordingly. 
 
\subsection{Global Positioning } 
The Global Positioning System or GPS is a satellite based technology which can locate a receiver
device anywhere on Earth to a very accurate degree. It works by having a receiver detecting at least
4 signals from a constellation of satellites in medium earth orbit. Each satellite transmits a 
signal containing the time that the signal was sent. Each satellite is moving at roughly 8700 miles per hour. 
The satellites have an extremely accurate clock on board. The clocks have relativistic correction factors for 
both gravitational and velocity effects. The position of each satellite is monitored by ground based radar stations 
to ensure that their positions are always well known. Initially, the system had a random error programmed into it. 
This feature was called 'selective availability', and was meant to be a deterrent to potential tracking threats. 
The result of selective availability was that the entire location grid was shifted by some random amount at a random time. 
the random dither function). The solution to the problem was to purchase a number of commercial GPS receivers for 
the troops, and remove the dither function from the system. The selective availability function was permanently removed 
in 2000. This is an example of managing errors both to reduce threats and to solve operational issues.  

\subsection{Offensive Cyber Security} 
Stuxnet was the most complex example of malware ever developed. It was originally discovered in June of 2010.
Its purpose was to disrupt the enrichment of uranium at the Natanz facility in Iran. It is understood that the software 
was developed by a state sponsored group, though the actual authors are a matter of speculation. The software gained entry
into a air-gapped network of high speed centrifuges controlled by a large array of Siemens S7 programmable logic
controllers. The software was designed to change the speed of individual centrifuges in ways that would cause early
failure. The effect of this was that scientists spent a great deal of time trying to debug and repair equipment rather 
than to spend time perfecting their enhrichment process. The aggregate enrichment rate fell to below the point of 
producing weapons grade uranium 235. One aspect of the operation of Stuxnet is that the presentation of system status
data to the monitor system was pre-recorded and looped to the monitor to cover up the fact that the system was 
manipulating the centrifuges behind the scenes. This is similar to the bank robbery scene the movie Ocean's 11.
This is an example of misrepresenting system status to make it look like the system is running error free. 


%\subsection{CIA Triad} 

\section{Modern Problems} 

\subsection{Artificial Intelligence} 
In the world of data security, there is a concept known as the CIA triad. This acronym is formed from 
Confidentiality, Integrity, and Availability. Confidentiality is the notion that data should only be accessed by 
authorized users and protected from unauthorized access. Integrity is the idea that data has not been tampered with or 
altered by unauthorized sources. Availability is a guarantee that data can be accessed when needed by authorized users. 
With recent advances in AI technology, we have to start thinking about a few new things.  
The xAi agent known as Grok, recently (early July, 2025) began exhibiting very different behavior in its responses to 
prompts. It started praising Hitler. This brings up troubling questions about how AI agents are responding. AI agents
typically use large language models which are trained with extreme amounts of mostly internet available data. 
We've seen several examples of how technologies use and react to error conditions. In the case of AI, there are 
two central issues.
\begin{itemize}
\item Does the training data contain errors, and what in fact, constitutes and error.
\item Can the agent adopt a point of view.
\end{itemize}
In the case of general AIs, it is certain that some of the training data has inaccuracies. In the specific case of 
Grok, it appears that its behavior changed its point of view over a very short period of time. This suggests that 
the point of view of an AI agent can be quickly manipulated. The existing notion of the CIA triad might benefit from
a new aspect, one of data validity. Though there are many complexities at hand, the idea of data validity in some
sort of metric, would enable a degree of accountability for AI agents. 
Count we adopt a 'peer review' model ? Could we learn from how accounts of history are documented? Could there
be an established corpus if data of known provenance? 
There are many examples where both media outlets and general internet information sources are distributing data that
is objectively false. This comes in the form of data meant to promote certain political points of view, or 
economic results, or even hate driven false narratives. In any case, it is important that we try to address this 
issue. There will come a time when the reasoning capacity of AI systems can rival or exceed that of our own human
capacity. Without a reasonable method to validate inputs to these systems, we can expect at best, flawed results, and
at worst, highly manipulated outcomes. Neither case is good. 
 

%------------------------------------------------



%----------------------------------------------------------------------------------------
%	 REFERENCES
%----------------------------------------------------------------------------------------

%\printbibliography % Output the bibliography

%----------------------------------------------------------------------------------------

\end{document}
